%HEADER: Where you make all of the settings for your document.

\documentclass[titlepage,12pt] {article}
\usepackage{setspace}  %allows the line spacing to be set
\usepackage{epsfig} %lets you put .eps figures into the document
\usepackage{amsbsy} %provides a command for bold math symbols
\usepackage{fancyhdr}
\usepackage{amsmath} %provides various features for writing math formulas
\usepackage{graphicx} %allows you to insert figures
\usepackage{natbib} %adds the bibliography
\usepackage{tabularx} %Allows you to make tables
\usepackage{longtable}
\usepackage{floatflt}
\usepackage{amssymb}
\usepackage{rotating}

%the following change margins and font size
\topmargin =0cm \evensidemargin = -0.06cm \oddsidemargin=-0.06cm
\textwidth=16.8cm \textheight=23cm
\def\baselinestretch{1.2}

%puts a header on all pages
%\pagestyle{fancy}

%Put and empty footer (or you'll have page #'s on the bottom too)
\fancyfoot{}
%This puts page number on right of header and title on left
%\fancyhead[LE, RO]{\bfseries\thepage}

%this is what the page header is
%\lhead{Summary Statistics}


%_________________________________________________________________________
\begin{document} %This starts the actual document.


\title{\emph{ADZE}: \\
Allelic Diversity Analyzer\\Version 1.0}
\author{Zachary A.~Szpiech\thanks{Comments on \emph{ADZE} can be sent to
szpiechz@umich.edu} \\
Bioinformatics Program \\
University of Michigan \and Mattias Jakobsson \\
Bioinformatics Program \\ Dept.~of Human Genetics \\ University of
Michigan \and Noah A.~Rosenberg \\ Bioinformatics Program\\ 
Dept.~of Human Genetics \\
University of Michigan }


\date{\today}

\maketitle

%_________________________________________________________________________

\tableofcontents

\newpage
\section{Introduction}



\section{Getting started}
Executables are available for \emph{ADZE} for Linux and Windows (source code 
available upon request).  The \emph{datafile} is expected to be in the same 
format as used by \emph{structure} (see section 
\ref{sec:datafile}).

\subsection{Availability}
\label{sec:avail}
Pre-compiled executables for Linux (32-bit and 64-bit) and Windows (32-bit 
only) are available at:\\
\ttfamily
http://rosenberglab.bioinformatics.med.umich.edu/adze.html
\rmfamily\\
Use the following citation for \emph{ADZE}.\\[.15cm]
Szpiech, Z.~A., Jakobsson, M., and Rosenberg, N.~A. (2007).
ADZE: Allelic Diversity Analyzer Version 1.0,
http://rosenberglab.bioinformatics.med.umich.edu/adze.html

\subsection{Installing \emph{ADZE}}
\subsubsection{Linux}
Open a terminal and move to the location of the \ttfamily .tar.gz \rmfamily 
file.  Extract it by typing: 
\ttfamily tar -xzvf adze-1.0-XX.tar.gz\rmfamily, where 
XX is the appropriate architecture. This will create a new directory called 
\ttfamily ADZE-1.0 \rmfamily containing the executable.

\subsubsection{Windows}
Extract the file \ttfamily adze-1.0-win32.zip\rmfamily.  This will create a 
directory called \ttfamily ADZE-1.0 \rmfamily containing the executable.

\subsection{Running \emph{ADZE}}
\label{sec:running}
When executing \emph{ADZE}, the \emph{datafile} specified in the 
\emph{paramfile} must be in the same directory as the \emph{ADZE} executable, 
or the whole path must be specified in the \emph{paramfile}.

\subsubsection{Linux}


\subsubsection{Windows}

\section{Input Files}
\label{sec:infile}

\subsection{\emph{datafile}}
\label{sec:datafile}

\section{Output Files}
\label{sec:infile}

\section{Usage options}
\label{sec:use}


\section{Command line arguments}
\label{sec:cmdline}
The command line flags are described below . 
\\
\\
--error <double>: The assumed genotyping error rate.
	Default: 0.00

--hap <string>: A hapfile with one row per individual,
	and one column per variant.
	Variants should be coded 0/1/-9.
	Default: __hapfile

--help <bool>: Prints this help dialog.
	Default: false

--ind <string>: An indfile containing population and individual IDs.
	One row per individual, formatted <popID> <indID>
	Default: __indfile

--kde-bw <double>: Manually set the bandwidth for the KDE of lod scores.
	By default, the nrd0 rule of thumb is used.
	Default: -1.00

--kde-points <int>: The number of equally spaced points at which to do KDE.
	Default: 512

--lod-cutoff <double>: For LOD based ROH calling, specify a single LOD score cutoff
	above which ROH are called in all populations.  By default, this is chosen
	automatically per population with KDE.
	Default: -999999.00

--lod-cutoff-file <string>: For LOD based ROH calling, specify a file with LOD score cutoffs
	above which ROH are called for each population.
	File format is <pop ID> <cutoff>.
	By default, these cutoffs are chosen automatically per population with KDE.
	Default: _none

--map <string>: A mapfile with one row per variant site.
	Formatted <chr#> <locusID> <genetic pos> <physical pos>
	Default: __mapfile

--max-gap <int>: A LOD score window is not calculated if the gap (in bps)
	between two loci is greater than this value.
	Default: 200000

--out <string>: The base name for all output files.
	Default: outfile

--raw-lod <bool>: If set, LOD scores will be output to gzip compressed files.
	Default: false

--resample <int>: Number of resamples for estimating allele frequencies.
	When set to 0 (default), rohscan will use allele
	frequencies as calculated from the data.
	Default: 0

--size-bounds <double1> ... <doubleN>: Specify the short/medium and medium/long
	ROH boundaries.  By default, this is chosen automatically
	with a 3-component GMM.  Must provide 2 numbers.
	Default: -1.000000

--size-bounds-file <string>: A file speficying the short/medium and medium/long
	ROH boundaries per population.
	File format <pop ID> <short/medium boundary> <medium/long boundary>
	By default, this is chosen automatically
	with a 3-component GMM.  Must provide 2 numbers.
	Default: _none

--tfam <string>: A tfam formatted file containing population and individual IDs.
	Default: __tfamfile

--threads <int>: The number of threads to spawn during calculations.
	Default: 1

--tped <string>: A tped formatted file containing map and genotype information.
	Default: __tpedfile

--tped-missing <string>: Missing data code for TPED files.
	Default: 0

--winsize <int>: The window size in # of SNPs in which to calculate LOD scores.
	Default: 60


\section{Output files}
\label{sec:outfile}


\end{document}
